\documentclass{beamer}
 
\usepackage[utf8]{inputenc}
\usepackage{tabto}
 
 
\title{IP Major}
\author{Wannes Fransen \& Tom Eversdijk}
\institute{UC Leuven}
\date{2020}
 
 
 
\begin{document}

\frame{\titlepage}


\begin{frame}
    \frametitle{Who we are}
    \begin{itemize}
        \item Wannes Fransen
        \item Tom Eversdijk
    \end{itemize}
\end{frame}


\begin{frame}
    \frametitle{Classes - contacturen}

    \begin{itemize}
        \item 11/12 * 2hrs = 24hrs
        \item 8/9 * 1hrs = 9hrs (support - nonobligatory)
        \item 3stp = 90hrs
    \end{itemize}

\end{frame}

\begin{frame}
    \frametitle{Evaluation}

    \structure{Project}
    \begin{itemize}
        \item Assignment \tab 30 \%
        \item Exam  \tab 70 \%
    \end{itemize}

    \vfill

    \structure{Requirements}
    \begin{itemize}
        \small
        \item The requirement to get a "tolereerbaar" result,
         is to correctly implement the required features at the exam.
         \item Following requirements (see ECTS) to participate at the exam are applicable:
         \begin{itemize}
             \item You must submit your assignment before the deadline
             \item Your submitted assignment must implement the required features in order to participate at the exam
         \end{itemize}
    \end{itemize}
\end{frame}


\begin{frame}
    \frametitle{How this course is structured}

    \begin{itemize}
        \item Classes are theory / small demonstrations \& exercises, with some time for questions
        \item You work independently, questions are asked in the support hours
        \item No questions through mail
    \end{itemize}
\end{frame}

\begin{frame}
    \frametitle{IP major - purpose}

    \begin{itemize}
        \item Concepts \& purposes of a framework
        \item Use of generators
        \item Using a CI/CD platform with a non-standard language
        \item Implementing advanced features with the least possible effort
        \item Writing easy to maintain code
        \item ...
    \end{itemize}

\end{frame}

\begin{frame}
    \frametitle{Language}

    Elixir
    \begin{itemize}
        \item Elrang VM
        \item Functional
        \item Process-oriented
        \item Fault-tolerant \href{https://stackoverflow.com/questions/8426897/erlangs-99-9999999-nine-nines-reliability}{[LINK]}
    \end{itemize}

\end{frame}

\begin{frame}
    \frametitle{Quote uptime}

    \begin{quote}
        

    The AXD301 has achieved a NINE nines reliability (yes, you read that right, 
    99.9999999\%). Let’s put this in context: 5 nines is reckoned to be good (5.2 minutes of downtime/year). 7 nines almost unachievable ... but we did 9.

    Why is this? No shared state, plus a sophisticated error recovery model.

    \end{quote}

\end{frame}

\begin{frame}
    \frametitle{Companies using the BEAM}

    \begin{itemize}
        \item WhatsApp
        \item Pinterest
        \item Discord \href{https://blog.discordapp.com/how-discord-handles-two-and-half-million-concurrent-voice-users-using-webrtc-ce01c3187429}{[LINK]}
        \item IBM Cloudant
        \item AdRoll
        \item Bet365
        \item ...
    \end{itemize}

\end{frame}

\end{document}

